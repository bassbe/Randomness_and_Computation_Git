\documentclass{article}

\usepackage{fancyhdr}
\usepackage{extramarks}
\usepackage{amsmath}
\usepackage{amsthm}
\usepackage{amsfonts}
\usepackage{tikz}
\usepackage[plain]{algorithm}
\usepackage{algpseudocode}

\usetikzlibrary{automata,positioning}

%
% Basic Document Settings
%

\topmargin=-0.45in
\evensidemargin=0in
\oddsidemargin=0in
\textwidth=6.5in
\textheight=9.0in
\headsep=0.25in

\linespread{1.1}

\pagestyle{fancy}
\lhead{\hmwkAuthorName}
\chead{\hmwkClass\ : \hmwkTitle}
%\rhead{\firstxmark}
\lfoot{\lastxmark}
\cfoot{\thepage}

\renewcommand\headrulewidth{0.4pt}
\renewcommand\footrulewidth{0.4pt}


\setlength\parindent{0pt}

%make a line command available
%\noindent\rule{8cm}{0.4pt} // will make a  line 8cm long
%\noindent\makebox[\linewidth]{\rule{\paperwidth}{0.4pt}}

\newcommand{\RN}[1]{%
  \textup{\uppercase\expandafter{\romannumeral#1}}%
}
%\RN{<input number>}


%
% Create Problem Sections
%

\newcommand{\enterProblemHeader}[1]{
    \nobreak\extramarks{}{Problem \arabic{#1} continued on next page\ldots}\nobreak{}
    \nobreak\extramarks{Problem \arabic{#1} (continued)}{Problem \arabic{#1} continued on next page\ldots}\nobreak{}
}

\newcommand{\exitProblemHeader}[1]{
    \nobreak\extramarks{Problem \arabic{#1} (continued)}{Problem \arabic{#1} continued on next page\ldots}\nobreak{}
    \stepcounter{#1}
    \nobreak\extramarks{Problem \arabic{#1}}{}\nobreak{}
}

\setcounter{secnumdepth}{0}
\newcounter{partCounter}
\newcounter{homeworkProblemCounter}
\setcounter{homeworkProblemCounter}{1}
\nobreak\extramarks{Problem \arabic{homeworkProblemCounter}}{}\nobreak{}

%
% Homework Problem Environment
%
% This environment takes an optional argument. When given, it will adjust the
% problem counter. This is useful for when the problems given for your
% assignment aren't sequential. See the last 3 problems of this template for an
% example.
%
\newenvironment{homeworkProblem}[1][-1]{
    \ifnum#1>0
        \setcounter{homeworkProblemCounter}{#1}
    \fi
    \section{Problem \arabic{homeworkProblemCounter}}
    \setcounter{partCounter}{1}
    \enterProblemHeader{homeworkProblemCounter}
}{
    \exitProblemHeader{homeworkProblemCounter}
}

%
% Homework Details
%   - Title
%   - Due date
%   - Class
%   - Section/Time
%   - Instructor
%   - Author
%

\newcommand{\hmwkTitle}{Randomness and Computation}
\newcommand{\hmwkDueDate}{29 April 2015}
\newcommand{\hmwkClass}{}
\newcommand{\hmwkClassTime}{}
\newcommand{\hmwkClassInstructor}{Professor Jose Bento}
\newcommand{\hmwkAuthorName}{Benjamin Bass \& Roseanna Chu}

%
% Title Page
%

\title{
    \vspace{2in}
    \textmd{\textbf{\hmwkClass\ \hmwkTitle}}                                  \\
    \normalsize\vspace{0.1in}\small{Lecture\ from:\ \hmwkDueDate}             \\
    \vspace{0.1in}\large{\textit{\hmwkClassInstructor\ \hmwkClassTime}}
    \vspace{3in}
}

\author{\textbf{\hmwkAuthorName}}
\date{}

\renewcommand{\part}[1]{\textbf{\large Part \Alph{partCounter}}\stepcounter{partCounter}\\}

%
% Various Helper Commands
%

% Useful for algorithms
\newcommand{\alg}[1]{\textsc{\bfseries \footnotesize #1}}

% For derivatives
\newcommand{\deriv}[1]{\frac{\mathrm{d}}{\mathrm{d}x} (#1)}

% For partial derivatives
\newcommand{\pderiv}[2]{\frac{\partial}{\partial #1} (#2)}

% Integral dx
\newcommand{\dx}{\mathrm{d}x}

% Alias for the Solution section header
\newcommand{\solution}{\textbf{\large Solution}}

% Probability commands: Expectation, Variance, Covariance, Bias
\newcommand{\E}{\mathrm{E}}
\newcommand{\Var}{\mathrm{Var}}
\newcommand{\Cov}{\mathrm{Cov}}
\newcommand{\Bias}{\mathrm{Bias}}

\begin{document}

\maketitle
    
\pagebreak
We have
$
\mathbb{E}(X|Y) = z?                                                          \\
\text{ if } 
\mathbb{E}(|x|) < \infty
$

\begin{enumerate}
\item $\mathbb{F}_{z} \subset \mathbb{F}_{y}$
\item $\forall A \in \mathbb{F}_{Y}, \mathbb{E}(\RN{2}_{A}z) = \mathbb{E}(\RN{2}_{A}X)$
\end{enumerate}
Note, that $$\mathbb{E}(X|A) = \dfrac{\mathbb{E}(x\RN{2}_{A})}{\mathbb{P}(A)}
$$

\noindent\rule{15cm}{0.4pt}%-----------------------------------------------------

Compatible with the previous definition, when y is discrete. Where 
$
\mathbb{E}(|X|) < \infty \text { then, } (w) = \mathbb{E}(X|\{y = y_{i}\}) 
\text{ and } y_{i} : y(w) = y_{i}
$

\noindent\rule{15cm}{0.4pt}%-----------------------------------------------------

\begin{itemize}
\item $ \mathbb{E}(X+Y|Z) = \mathbb{E}(X|Z) + \mathbb{E}(Y|Z)  $
\item $ \mathbb{E}(aX|Y) = a\mathbb{E}(X|Y)                    $
\item $ \mathbb{E}(P(y)X|Y) = f((y)\mathbb{E}(X|Y))            $
\item $ \mathbb{E}(\mathbb{E}(X|Y)) = \mathbb{E}(X)            $
\end{itemize}

\noindent\rule{15cm}{0.4pt}%-----------------------------------------------------

\begin{equation*}
  \begin{split}
    \mathbb{E}(X) &= \mathbb{P}(y = y_{1})\mathbb{E}(X|Y=Y_{1})               \\
                  &= + \mathbb{P}(y = y_{2})\mathbb{E}(X|Y=Y_{2})             \\
                  &= + \mathbb{P}(y = y_{3})\mathbb{E}(X|Y=Y_{3})             \\
                  &= + ...                                                    \\
  \end{split}
\end{equation*}

\noindent\rule{15cm}{0.4pt}%-----------------------------------------------------

x If x8y are jointly continuous.                                              \\
$f_{y}(y \neq 0)$ and if $\mathbb{E}(|X|) < \infty$, then 
$\mathbb{E}(X|Y) = G(y)$                                                      \\

$$G(y) = \int_{-\infty}^{\infty} x\dfrac{f_{xy}(x,y)}{f_{y}(y)} dx$$

which we arrived at from

\begin{equation*}
  \begin{split}
    \mathbb{P}(Y \in A) &= 
    \mathbb{P}\big( (x,y) \in R \times A\big)                                 \\
                        &= 
    \int_{-\infty}^{\infty}\int_{A}^{} f_{xy}(x,y) dydx                       \\
    f_{y}(y)            &=
    \int_{-\infty}^{\infty} f_{xy}(x,y) dx                                    \\
  \end{split}
\end{equation*}

\noindent\rule{15cm}{0.4pt}%-----------------------------------------------------

$\underline{Exercise:}$                                                       \\
\begin{displaymath}
   f_{xy}(x,y) = \left\{
     \begin{array}{lr}
       k(x+y) & : x,y \in [0,1]\\
       0 & : otherwise
     \end{array}
   \right.
\end{displaymath} 
$\underline{Answer:}$
From the first property, we use the theorem:                                  \\

Th: if $ F_{z} \subset F_{y}$\\
then $z = f(y)$ for some measurable function $f$.\\
\\
%All questions that we can ask about $z$ are contained in $f(y)$, then there exists a mapping between $$\\
\\
If the questions that $z$ generates are contained in $f(y)$, then points mapped previously to the arrival space  may be mapped to the same point.\\
\begin{enumerate}
\item This implies $ \mathbb{E}(X|Y) = G(y)$
\item if true for $\forall a \in F_{y}$ then let 
$A = \{ y \in [0,a]\} \in F_{y}$ then $\mathbb{E}(\RN{2}_{A}z)
 = \mathbb{E}(\RN{2}_{A}x)$ must be true and equal to 
$$\mathbb{E}(\RN{2}_{Y \in [0,x]}G(y)) = \mathbb{E}(\RN{2}_{Y\in[0,x]}X)$$    \\
\end{enumerate}


\noindent\rule{15cm}{0.4pt}%-----------------------------------------------------

Now we can start doing some calculations. Lets calculate $f_{y}$.             \\
$$f_{y}(y) = \int_{-\infty}^{\infty} f_{xy}(x,y) dx =  
 k(\frac{1}{2} + y), y \in [0,1]$$                                            \\
\begin{equation*}
  \begin{split}
    \int_{-\infty}^{\infty} \RN{2}_{y} \in [a,a] \cdot G(y)f_{y}(y)dy 
     &= \int_{0}^{x} G(y) k(\frac{1}{2} +y) dy
  \end{split}
\end{equation*}
This is the same an important equivalence:
$$ \mathbb{E}(\RN{2}_{A}y\in[0,x] G(y)) 
= \mathbb{E}(\RN{2}_{y\in[0,a]}x)$$                                \\

is equivalent to:

$$\int_{-\infty}^{\infty}\int_{-\infty}^{\infty} \RN{2}_{y\in [0,a]} xf_{xy}(x,y)dxdy 
= \int_{0a}^{a}\int_{0}^{1} xk(x+y) dxdy$$\\

Thus:

$$Eq\RN{1} = Eq\RN{2} \forall a \implies \frac{d}{da}\RN{1} = \frac{d}{da}\RN{2}$$

\begin{equation*}
  \begin{split}
    G(y) k(\frac{1'}{2} + y) &= \int_{0}^{1} k(x+y) dx                        \\
                             &= k(\frac{1}{3} + \frac{1}{2}y)
  \end{split}
\end{equation*}

From which we can conclude:                                                   \\
\\
\begin{equation*}
  \begin{split}
G(y) &= \dfrac{\frac{1}{3} + \frac{1}{2}y}{\frac{1}{2} + y}                   \\
\mathbb{E}(X|Y)     &= \dfrac{\frac{1}{3} + \frac{1}{2}y}{\frac{1}{2} + y}
  \end{split}
\end{equation*}



\noindent\rule{15cm}{0.4pt}%-----------------------------------------------------

If we use the formula we have, we can check the results we have.\\
\\
Formula: \\
$$G(y) = \int_{-\infty}^{\infty} \dfrac{f_{xy}(x,y)}{f_{y}(y)} dx$$
\\
Confirm results:\\
\\
\begin{equation*}
  \begin{split}
    G(y) &= \int_{-\infty}^{\infty} x\dfrac{f_{xy}(x,y)}{f_{y}(y)} dx          \\
         &= \dfrac{\int_{0}^{} k(x+y)xdx}{\int_{-\infty}^{\infty} kx(x+y) dx} \\
         &= \dfrac{\frac{1}{3} + \frac{1}{2}y}{\frac{1}{2} + y}            \\
  \end{split}
\end{equation*}

\noindent\rule{15cm}{0.4pt}%-----------------------------------------------------

\end{document}

%%% Local Variables:
%%% mode: latex
%%% TeX-master: t
%%% End:
